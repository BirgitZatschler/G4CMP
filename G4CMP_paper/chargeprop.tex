At present, the charge transport component of the G4CMP framework only includes
transport in germanium. For charge carrier transport in germanium, there are
two processes to consider: acceleration by an applied electromagnetic field,
and emission of Neganov-Luke phonons.

When an incoming particle scatters in the germanium crystal, electron-hole
pairs are produced. In the example of the CDMS dark matter detectors, the
charge carriers are drifted by an external electric field to the surfaces of
the detector. 

\subsection{Neganov-Luke Phonons}
As the charge carriers are accelerated through the crystal, they emit phonons
in a process that is analogous to Cerenkov radiation.
\subsubsection{Holes}
Charge carrier-hole scattering is an elastic process, conserving energy and
momentum.

\begin{figure}[htpb]
    \centering
    \includegraphics[width=.5\textwidth]{scatter}
    \caption{A charge carrier has initial wavevector $\vec{k}$ and emits a
    Neganov-Luke phonon with wavevector $\vec{q}$ \cite{Leman}}.
    \label{fig:scatter}
\end{figure}

From conservation of energy and momentum, $k'^2 = k^2 + q^2 - 2kq\cos{\theta}$
and $q = 2(k\cos{\theta} - k_L$, with $k_L$ defined as $k_L = mv_L/\hbar$,
where $v_L$ is the longitudinal phonon phase speed. Solving for $\phi$,
\begin{equation}
    \cos{\phi} = \frac{k^2 - 2k_s(k\cos{\theta} - k_s) - 2(k\cos{\theta} -
k_s)^2}{k\sqrt{k^2 - 4k_s(k\cos{\theta} - k_s})}
    \label{eq:scatterangle}
\end{equation}
Where $k_L = mv_L/\hbar$. Using Fermi's Golden Rule, we can determine a
scattering rate \cite{Leman},
\begin{equation}
    1/\tau = \frac{v_Lk}{3l_0k_L}\left(1-\frac{k_L}{k}\right)^3
    \label{eq:rate}
\end{equation}
With an angular distribution of,
\begin{equation}
    P(k,\theta) d\theta =
    \frac{v_L}{l_0}\left(\frac{k}{k_L}\right)^2\left(\cos{\theta}-\frac{k_L}{k}\right)^2\sin{\theta}d\theta
    \label{eq:ang-dist}
\end{equation}
Where $0\le\theta\le\arccos{k_L/k}<\pi/2$ and $l_0$ is a characteristic
scattering length defined as $l_0 = \frac{\pi\hbar^4\rho}{2m^3C^2}$ with C
being the deformation potential constant for Ge \cite{Leman}.

\subsubsection{Electrons}
The effective mass of the hole in germanium is a scalar, so its
propagation is simple. The electron, however, has a tensor effective mass.
\begin{figure}[htpb]
    \centering
    \includegraphics[width=.5\textwidth]{gebands}
    \caption{Conduction and valence bands in Ge. At present, the G4CMP
    framework only simulates electrons propagating through the L conduction
band. At sufficiently low temperature and applied field, this should match
reality well \cite{Leman}.}
    \label{fig:gebands}
\end{figure}

For a coordinate system with one axis aligned with the principle axis of the
conduction valley, the electron's equation of motion is,
\begin{equation}
    \frac{eE_i}{m_i} = \frac{dv_i}{dt}
    \label{el_eq_mtn}
\end{equation}

However, to simplify the electron propagation, we transform to a coordinate
system in which the constant energy surfaces are spherical. In that space,
$v_i^* = v_i/\sqrt{m_c/m_i}$, where $m_c$ is given by $3/m_c = 1/m_\parallel +
2/m_\perp$. And so,

\begin{equation}
    \frac{eE^*_i}{m_c} = \frac{dv_i^*}{dt}
    \label{el_eq_mtn1}
\end{equation}

Once the coordinate system is rotated into the conduction valley frame, a
Herring-Vogt transformation is then applied,
\begin{equation}
    T_{HV} = \left( \begin{array}{ccc}
                    \sqrt{\frac{m_c}{m_{\parallel}}} & 0 & 0 \\
                    0 & \sqrt{\frac{m_c}{m_{\perp}}} & 0 \\
                    0 & 0 & \sqrt{\frac{m_c}{m_{\perp}}}\end{array}\right)
    \label{HV}
\end{equation}

From this space, the same recipe that applied to holes for propagation and
Neganov-Luke phonon emission can be followed for electrons. One potential issue
is the back-transform into real space. Because the HV matrix is not unitary, it
wont conserve energy. To mitigate the issue, the phonon's momentum magnitude is
kept from the HV space and the back-transform is only used to determine the
angular distribution \cite{Leman}.



